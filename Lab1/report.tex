\documentclass{article}
\usepackage{mhchem}
\usepackage{tikz}
\usepackage{pgfplots}
\usepackage{filecontents}
\usepackage[T1]{fontenc}
\usepackage[utf8]{inputenc}
\usepackage[english, russian]{babel}


\title{Исследование явления осмоса на базе лабораторного стенда и расчёт зависимостей}
\author{Нестеров И.Д.}
\date{}

\begin{document}
    \selectlanguage{russian}
    \maketitle
    \tableofcontents
    \newpage

    \addcontentsline{toc}{section}{Введение}
    \section*{Введение}

        \hspace*{4mm}\textbf{\textit{Цель работы:}} Ознакомиться и научиться проводить исследования осмотических
        процессов в частности провести расчёты осмотического давления и осмотического потока, используя законы Фика.
        
        \addcontentsline{toc}{subsection}{Осмос}
        \subsection*{Осмос}
            \hspace*{4mm}\textit{Осмос} – частный случай \textit{диффузии}. Другими словами, это диффузия воды
            через полупроницаемую мембрану вниз по градиенту концентрации, когда
            растворенное вещество не может диффундировать через мембрану, а вода может,
            если мембрана проницаема для воды, но не для растворенного
            вещества, вода будет выравнивать свою собственную концентрацию путем
            диффундирования в сторону с более низкой концентрацией воды.

            \begin{itemize}
                \item Вода считается универсальным растворителем -
                она связывает и растворяет полярные или заряженные
                молекулы (растворённые вещества)

                \item Поскольку растворённые вещества не могут
                проникнуть через клеточную мембрану без посторонней
                помощи, вода будет перемещаться, чтобы уравнять оба
                раствора

                \item При более высокой концентрации растворённого
                вещества в растворе меньше свободных молекул воды,
                поскольку вода связана с растворённым веществом
            \end{itemize}

            % Не забудь вставить изображение, отображающее
            % процесс протекания осмоса

        \addcontentsline{toc}{subsection}{Осмотическое давление}
        \subsection*{Осмотическое давление}
            \hspace*{4mm}\textit{Осмотическое давление} можно определить как минимальное давление, которое необходимо
            приложить к раствору, чтобы остановить поток молекул растворителя через
            полупроницаемую мембрану (осмос). Это коллигативное свойство, которое
            зависит от концентрации частиц растворенного вещества в растворе. Поэтому, по
            Вант-Гоффу, для вычисления осмотического давления можно воспользоваться
            уравнением Менделеева-Клапейрона:

            $$P = \frac{m}{MV} \* RT = CRT$$

            где C – молярная концентрация растворенного вещества в растворе, R –
            универсальная газовая постоянная, Т – температура, m – масса растворенного
            вещества, V – объем раствора, M – молярная масса растворенного вещества.

        \addcontentsline{toc}{subsection}{Осмотические поток}
        \subsection*{Осмотический поток}
            \hspace*{4mm}Для расчёта потока используем стандартную формулу для диффузного потока, однако с учетом того, что осмос
            является односторонним процессом диффузии, где движущая жидкость является
            вода, интерпретируя закон Фика под эту цель.

            % Maybe, align?
            Первый закон Фика:
            $$J = -D\frac{dC}{dx}$$
            
    \addcontentsline{toc}{section}{Ход работы}
    \section*{Ход работы}

        \addcontentsline{toc}{subsection}{Растворы}
        \subsection*{Растворы}
            \hspace*{4mm}Было приготовлено два раствора \ce{CuSO4*5H2O} с разными концентрациями.
            Концентрация первого раствора составляла 10\%, второго - 20\%.


        \addcontentsline{toc}{subsection}{Установка}
        \subsection*{Установка}



        \addcontentsline{toc}{subsection}{Наблюдения}
        \subsection*{Наблюдения}
            \hspace*{4mm}После сбора установки и начала эксперимента каждые 5 минут производились измерения высоты
            столбцов с растворами. \\

            Динамика высоты жидкости в первом сосуде, в котором концентрация составляла 10\%: \\

            \begin{tikzpicture}
                \begin{axis}[scale only axis,
                        ylabel=$h$,
                        xlabel=$t$,
                        xmax=35,
                        ymin=15, ymax=17,
                        xtick={0,5,...,30},
                        ytick={15.2,15.4,...,16.6},
                        axis lines=middle,
                        grid=both   
                    ] 
                    \addplot[only marks] table [col sep=comma] {./data/experiment1.csv};
                \end{axis}
            \end{tikzpicture}

            Динамика высоты жидкости во втором сосуде, в котором концентрация составляла 20\%: \\

            \begin{tikzpicture}
                \begin{axis}[scale only axis,
                        ylabel=$h$,
                        xlabel=$t$,
                        xmax=35,
                        ymin=17, ymax=21,
                        xtick={0,5,...,30},
                        ytick={17.5,18.0,...,20.5},
                        axis lines=middle,
                        grid=both   
                    ] 
                    \addplot[only marks] table [col sep=comma] {./data/experiment2.csv};
                \end{axis}
            \end{tikzpicture}

    \addcontentsline{toc}{section}{Обработка данных}    
    \section*{Обработка данных}

    \addcontentsline{toc}{section}{Результаты и Выводы}
    \section*{Результаты и Выводы}


\end{document}